\section{Background}

In this section we present the background of NVMe interface and SPDK 
framework.

\subsection{NVMe and I/O Queue}

NVMe~\cite{nvme} is an open interface specification designed to allow host
software to communicate with a non-volatile memory subsystem (NVM) via a 
peripheral component interconnect express (PCIe) bus. Previous standards
such as serial-attacked SCSI and serial advanced technology attachment
can handle queue depths of 254 and 32 respectively. NVMe is able to 
handle queue depths of up to 65535 I/O queues with up to 64 Ki outstanding
commands per I/O queue, which allows an NVMe device to support parallel
operations. An I/O queue is composed of a submission queue and a completion
queue. Host software issues I/O commands to a submission queue and completions
are placed into the associated completion queue by the controller. Note that
the order of completions is not determined by the submission oder of the commands.

\subsection{SPDK and Block Devices}

SPDK~\cite{spdk} is an open source library that allows developer to implement
high performance, scalable, user-mode storage applications. A block device 
in SPDK is an abstraction of all block devices, where I/O commands are 
processed and issued to corresponding physical block devices such as NVMe 
devices and Malloc devices. SPDK provides a event framework, where  
different threads to exchange data through passing messages to one another.
It allows a user to build asynchronous, lockless, and high performance
applications.

\begin{figure}[t]
  \includegraphics[width=\linewidth]{spdk_arch}
  \caption{Example of an SPDK application.}
  \label{fig:spdk_arch}
\end{figure}

For each thread, SPDK uses an io channel to represent the channel for accessing an 
I/O device. I/O requests issued to the block device will be forwarded to the underlying
physical device. In our implementation, io channel corresponds to I/O queue 
of the underlying NVMe device, the framework builds I/O commands based on I/O 
requests and submits them to submission queue. It then polls for I/O completion
on each queue pair with outstanding I/O to receive completion callbacks.  
Figure~\ref{fig:spdk_arch} provides a graphical
representation of a host application using SPDK block devices to interact with
an NVMe device. In the host application, each thread submit I/O requests to its
corresponding I/O channel and the I/O channel forward the I/O requests to the 
actual physical device based on the implementation of the physical block device.
The framework invokes the callback function when the I/O request is finished.

\subsection{TestFS}
TestFS\cite{testfs} is a user space file system which is similar to EXT3, a
journaled file system that is commonly used by the Linux kernel. TestFS
has three levels of indirections. A super block points to meta data and inode 
block for root directory. Inode blocks point to indirect inode blocks or 
data blocks.