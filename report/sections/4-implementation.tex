\section{Implementation Details}
To implement asynchronous writes with multiple threads in testFS, we added
utilities to facilitate thread synchronization and we modified the file
system's write path to make it more amenable to asynchronous writes. For a
mapping of the key changes to the relevant source code files, please see
Appendix~\ref{apx:codewalkthrough}.

\subsection{Thread Synchronization}\label{sec:threadsync}
In our multi-threaded architecture, the control thread needs to be able to wait
for asynchronous I/O requests to complete. We accomplish this through thread
synchronization using shared objects: a primitive
future~\cite{futures-halstead85} or several
semaphores~\cite{semaphores-dijkstra65}. We describe both approaches, however
for the remainder of the report we only reference the primitive future
technique as the underlying ideas are similar.

\subsubsection{Primitive Futures}\label{sec:futures}
We implemented a shared object that we call a future, similar in spirit to
futures used in other concurrent programming
languages~\cite{futures-halstead85}. Our future is a shared {\tt struct} that
contains monotonically increasing counter variables---one per thread.
Listing~\ref{lst:future} shows the definition of our future.

\begin{lstlisting}[language=C,
  caption={The definition of our future.},
  captionpos=b,
  label={lst:future}]
struct future {
  volatile size_t counts[NUM_THREADS];
  size_t expected_counts[NUM_THREADS];
};
\end{lstlisting}

To use a future, the caller of an asynchronous function will create a future
and then pass a pointer to it to the asynchronous function. The key idea is
that {\tt expected\_counts} keeps track of the number of issued asynchronous
requests and that {\tt counts} keeps track of the number of completed
asynchronous requests. The {\tt counts[i]} variable is incremented when an
asynchronous request completes on thread {\tt i}. A thread waits for a future's
asynchronous operations to complete by spinning on the counters---essentially
waiting for {\tt expected\_counts[i] == counts[i]} for all {\tt i}. We defined
a helper function called {\tt spin\_wait()} to do this; it takes a pointer to a
future and spins on the counters.

\paragraph{Example.}
Suppose the control thread wants to call an asynchronous function that should
be handled by thread {\tt i}. The control thread will create a future and then
increment {\tt expected\_counts[i]}. It then passes a pointer to the future to
the asynchronous function. When the asynchronous operation completes on thread
{\tt i}, thread {\tt i} will increment {\tt counts[i]}.

\paragraph{Avoiding Locks by Design.}
Despite the presence of shared data, locks are not needed to use the future.
The {\tt counts} variable does not need to be guarded by locks because there is
one counter per thread, meaning only one thread writes to a given counter. The
{\tt expected\_counts} variable is not shared; it is only used by the
asynchronous function's calling thread. The {\tt counts} variable is marked
as {\tt volatile} to ensure the compiler does not generate code that caches the
counters in a register.

\subsubsection{Semaphores}
Another approach for thread synchronization is to use semaphores---creating one
per thread. The value of the semaphore represents the number of completed
asynchronous requests on a given thread. Each time an asynchronous request
completes on thread {\tt i}, the thread will post to semaphore {\tt i}. To wait
for asynchronous requests to complete, the control thread waits on the
semaphores of the relevant worker threads.

\paragraph{Avoiding Lock Contention.} We create one semaphore per thread to
avoid lock contention when modifying the semaphore's internal state. This
ensures only one thread (the metadata or data thread) is posting to a given
semaphore and only one thread (the control thread) is waiting on a given
semaphore.

\subsection{Asynchronous I/O}
\paragraph{Interface.}
To implement asynchronous I/O, we added asynchronous analogues of the {\tt
read\_blocks()} and {\tt write\_blocks()} functions called {\tt
read\_blocks\_async()} and {\tt write\_blocks\_async()}. These {\tt async}
functions accept the same arguments as their synchronous counterparts. However
callers of these {\tt async} functions also need to pass in a pointer to a
future and the thread ID on which the I/O request should be handled. This
allows the control thread to direct metadata block requests and data block
requests to distinct threads.

\paragraph{Implementation.}
When these asynchronous read/write functions are called, the future's {\tt
expected\_count} for the handling thread is first incremented and then the
request is sent to the handling thread using SPDK's message passing framework.
Upon receiving the request, the handling thread then submits the I/O request
to the underlying NVMe device using its dedicated I/O channel. When the request
completes, a callback is executed on the handling thread. The callback function
increments the future's {\tt count} for the handling thread to ``notify'' the
calling thread that the request has completed. Since the I/O request is
asynchronous, the calling thread is free to do other work while the I/O request
is being handled. If the calling thread needs to wait for the I/O request to
be completed it can call {\tt spin\_wait()} on the future.

\paragraph{Synchronous Baseline.}
To be able to compare asynchronous I/O requests and synchronous I/O
requests, we modified {\tt read\_blocks()} and {\tt write\_blocks()} so that
they can work with SPDK as well. These functions are implemented by calling
their asynchronous counterparts and then immediately waiting on the future.
This ensures that the I/O request completes before the function returns.

\subsection{Write Path Modifications}\label{sec:writepath}
\paragraph{Motivation.}
When we initially added asynchronous operations to file writes in testFS we
discovered a number of inefficiencies in the code that led to repeated reads
and writes of the same physical blocks. In the existing implementation:
\begin{enumerate*}[label={(\roman*)}]
  \item an entire bitmap block is flushed to the underlying device each time a
    single block is allocated;
  \item entire checksum blocks are flushed for each modified checksum;
  \item an entire inode block is flushed for each modified inode; and
  \item inode indirect blocks are read, modified, and flushed to the underlying
    device for every block appended to the file.
\end{enumerate*}

A write transaction in testFS may consist of writing multiple blocks to more
than one file and a single metadata block may contain metadata for multiple
data blocks. Therefore issuing entire metadata block writes for each modified
data block results in unnecessary write amplification.  We discovered that
these inefficiencies led to poor performance even in our asynchronous
implementation. This led us to reimplement the file write path in testFS.

\paragraph{Bulk Metadata Flushes.}
We made the observation that metadata such as block checksums, the data
block freemap, and ``opened'' inodes are cached in-memory. As a result, it is
unnecessary to flush modified metadata blocks to the underlying device each
time a new data block is appended to a file. We modified the file write
implementation to instead keep track of the dirty cached metadata blocks and
then to flush them all together in bulk at the end of a transaction. This
approach ensures that a metadata block is written to the underlying device at
most once per transaction.

\paragraph{In-Memory Indirect Blocks.}
We reduced the number of extra reads and writes of an inode indirect block by
keeping a copy of the indirect block in memory. The indirect block is loaded
into memory lazily (i.e. when it is first requested). This approach allows
subsequent reads and writes of the indirect block to occur in-memory. The
modified indirect block is flushed to the underlying device whenever the
in-memory copy of the inode is flushed.
