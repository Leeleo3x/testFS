\begin{abstract}
Non-volatile memory express (NVMe) based solid state devices have undergone
tremendous innovation, leading to significant increases in performance.
However file systems have been slow to keep up; in many cases their design
impedes their ability to leverage the full performance offered by NVMe storage
devices. For example, many file systems today:
\begin{enumerate*}[label={(\roman*)}]
  \item reside in kernel space, which results in needing slow context switches
    when interacting with the file system;
  \item use locks to coordinate access to shared data, which limits their
    scalability;
  \item are not written to take advantage of the parallelism exposed by NVMe.
\end{enumerate*}
How should next generation file systems be designed to leverage the performance
offered by NVMe storage devices?

In our project, we take a first step towards answering this question by
extending an existing user space file system to support NVMe storage devices by
using SPDK---a user space, polled, lockless NVMe device driver. The key idea
behind our design is to use a multi-threaded architecture to be able to submit
asynchronous I/O requests for different types of file system blocks (e.g.
metadata versus data) in parallel.

Through experiments, we show that our multi-threaded asynchronous write
path is able to offer up to a $1.5\times$ improvement for single file writes
and up to a $3\times$ improvement for multi-file single transaction writes when
compared to a write path implemented with synchronous I/O.
\end{abstract}
