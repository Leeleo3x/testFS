\section{Reproducing our Results}
Our code is available on GitHub at \url{https://github.com/Leeleo3x/testFS}.
To build and run the code, you need to first install {\tt cmake}. Then
follow the steps in the code repository readme to compile and run testFS.

To run our experiments, run the {\tt run-experiments} command on the testFS
command prompt. This command will run all the experiments we used to create the
charts in this project report. The experiments will take roughly 30 minutes to
complete. The raw data will be saved in {\tt .csv} files in the same directory
as where the testFS binary is stored.

Individual benchmarks can be executed with the {\tt bench} command on the
testFS command prompt. The {\tt bench} command is invoked with a benchmark name
and several arguments that are specific to the benchmark (e.g. {\tt bench
  raw\_seq\_read 1 1}). Table~\ref{tbl:bench-commands} lists the available
benchmarks and their arguments.

\begin{table}[h!]
  \centering
  \caption{The available benchmarks and their arguments.}
  \label{tbl:bench-commands}
  \begin{tabular}{c|c}
    {\bf Benchmark} & {\bf Arguments} \\ \hline
    {\tt raw\_seq\_read} & {\tt <trials> <blocks>} \\ \hline
    {\tt raw\_seq\_write} & {\tt <trials> <blocks>} \\ \hline
    {\tt e2e\_write} & {\tt <file> <trials> <\# files>}
  \end{tabular}
\end{table}

The {\tt raw\_seq\_read} and {\tt raw\_seq\_write} benchmarks perform raw
sequential reads/writes of a specified number of blocks. The {\tt e2e\_write}
benchmark writes data to a specified number of files. The data written to the
file is read from a file in the host operating system's file system. All three
benchmarks can be repeated for a specified number of trials.
