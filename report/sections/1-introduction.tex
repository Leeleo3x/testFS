\section{Introduction}
The emergence of the non-volatile memory express (NVMe) specification coupled
with faster solid state storage devices (SSDs) have together paved the way for
new opportunities to improve the performance of storage applications such as
file systems. In particular, the NVMe specification opens up new opportunities
for increased parallelism by giving applications parallel access to the
underlying storage device through multiple I/O queues.

Unfortunately, many file systems today are not designed to take advantage of
these opportunities. Many file systems today
\begin{enumerate*}[label={(\roman*)}]
  \item reside in kernel space, which results in needing slow context switches
    when interacting with the file system;
  \item use locks to coordinate access to shared data, which limits their
    scalability;
  \item are not written to take advantage of the parallelism exposed by NVMe.
\end{enumerate*}
How should next generation file systems be designed to leverage the potential
performance improvements offered by NVMe storage devices?

We take a first step towards answering this question by extending an existing
user space file system to support NVMe storage devices with the goal of
quantifying the potential performance benefit of NVMe-designed file systems.
Specifically we integrate testFS (a toy user space file system) with SPDK (a
user space lockless driver for NVMe devices) and we modify the file system's
write path to support multi-threaded asynchronous I/O as a proof of concept.

The key idea in our design is to use a multi-threaded architecture to allow
different types of I/O requests to be submitted to and queued on the underlying
device in parallel. This approach enables us to improve the performance of the
file system's write path by submitting asynchronous I/O requests for file
system metadata (e.g. inodes) and data in parallel by using different threads.

We benchmark our implementation and show that it can offer up to a $1.5\times$
improvement on single file writes and up to a $3\times$ improvement on
multi-file single transaction writes when compared to synchronous
single-threaded I/O.

\vspace{1em}
\noindent
In summary, our project makes the following contributions:
\begin{itemize}
  \item We propose a method to support multi-threaded asynchronous read and
    write requests on an NVMe SSD that can be implemented alongside existing
    synchronous code.
  \item We implement a proof of concept of our design by integrating testFS
    with SPDK: a user space, lockless, polled driver for NVMe devices.
  \item We benchmark our implementation and show that our design offers up to a
    $1.5\times$ improvement for single file writes and up to a $3\times$
    improvement for multi-file single transaction writes when compared to
    synchronous single-threaded I/O.
\end{itemize}
