\section{Introduction}
The emergence of the non-volatile memory express (NVMe) specification coupled
with faster solid state storage devices (SSDs) have together paved the way for
new opportunities to improve the performance of storage applications. In
particular, the NVMe specification introduces new avenues for parallelism by
providing applications parallel access to the underlying storage device through
multiple I/O queues. File systems, a major storage application, are prime
candidates for these performance improvement opportunities.

Unfortunately, many file systems today are not designed to take advantage of
these opportunities. They
\begin{enumerate*}[label={(\roman*)}]
  \item reside in kernel space, requiring slow context switches to access;
  \item use locks to coordinate access to shared data, limiting their
    scalability; and
  \item are not written to take advantage of the parallelism exposed by NVMe.
\end{enumerate*}
So how should next generation file systems be designed to leverage the
performance improvements offered by NVMe storage devices?

In this work we take a first step towards answering this question by extending
an existing user space file system to support NVMe storage devices. We do this
with the goal of {\it quantifying} the potential performance benefit of
NVMe-designed file systems. Specifically, as a proof of concept, we integrate
testFS (a toy user space file system) with SPDK (a user space lockless driver
for NVMe devices) and we modify the file system's write path to support
multi-threaded asynchronous I/O.

The key idea in our design is to use a multi-threaded architecture to allow
different types of I/O requests to be submitted to and queued on the storage
device in parallel. This approach enables us to improve the performance of the
file system's write path by submitting asynchronous I/O requests for file
system metadata (e.g. inodes) and data in parallel by using different threads.

We benchmark our implementation and show that it can offer up to a $1.5\times$
improvement on single file writes and up to a $3\times$ improvement on
multi-file single transaction writes when compared to synchronous I/O.

\vspace{0.75em}
\noindent
In summary, our project makes the following contributions:
\begin{itemize}
  \item We propose a method to support multi-threaded asynchronous I/O on an
    NVMe SSD that can be implemented alongside existing synchronous code.
  \item We implement a proof of concept of our design by modifying the testFS
    write path and integrating it with with SPDK.
  \item We benchmark our implementation and show that our design offers up to a
    $1.5\times$ improvement for single file writes and up to a $3\times$
    improvement for multi-file single transaction writes when compared to
    synchronous I/O.
\end{itemize}
