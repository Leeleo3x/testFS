\section{Conclusion}
The goal of this work was to explore the feasibility of using SPDK to interact
with NVMe as well as to quantify the potential performance benefits of a file
system designed for NVMe-based devices. To that end, we extended testFS---a
user space file system---to support NVMe devices by using SPDK. To leverage the
parallelism offered by NVMe devices, our key idea was to adopt a multi-threaded
architecture to be able to submit asynchronous I/O requests for different types
of data (e.g. inodes versus data) in parallel by using distinct threads. We
used this design to modify the testFS write path. Our experiments showed that
our design offered up to a $1.5\times$ improvement for single file writes and
up to a $3\times$ improvement for multi-file writes when compared to a write
path that used synchronous I/O. Our hope is that this work helps inspire the
design of the next generation of file systems that are specifically designed to
leverage the performance offered by NVMe devices.
